%% To submit your paper:
\documentclass[draft]{agujournal2019}
\usepackage{url} %this package should fix any errors with URLs in refs.
\usepackage{lineno}
\usepackage[inline]{trackchanges} %for better track changes. finalnew option will compile document with changes incorporated.
\usepackage{soul}
\linenumbers
%%%%%%%
% As of 2018 we recommend use of the TrackChanges package to mark revisions.
% The trackchanges package adds five new LaTeX commands:
%
%  \note[editor]{The note}
%  \annote[editor]{Text to annotate}{The note}
%  \add[editor]{Text to add}
%  \remove[editor]{Text to remove}
%  \change[editor]{Text to remove}{Text to add}
%
% complete documentation is here: http://trackchanges.sourceforge.net/
%%%%%%%

\draftfalse

\journalname{Geophysical Research Letters}


\begin{document}

%% ------------------------------------------------------------------------ %%
%  Title
%
% (A title should be specific, informative, and brief. Use
% abbreviations only if they are defined in the abstract. Titles that
% start with general keywords then specific terms are optimized in
% searches)
%
%% ------------------------------------------------------------------------ %%



\title{Duration of Individual Relativistic Electron Microbursts: A Probe Into Their Scattering Mechanism}

\authors{M. Shumko\affil{1, 2}, L.W. Blum\affil{3}, and A.B. Crew\affil{4}}

\affiliation{1}{NASA's Goddard Space Flight Center, Greenbelt, Maryland, USA}
\affiliation{3}{Universities Space Research Association, Columbia, Maryland, USA}
\affiliation{3}{University of Colorado Boulder, Boulder, Colorado, USA}
\affiliation{2}{Johns Hopkins University Applied Physics Laboratory, Laurel, Maryland, USA}

\correspondingauthor{M. Shumko}{msshumko@gmail.com}

%% Keypoints, final entry on title page.

%  List up to three key points (at least one is required)
%  Key Points summarize the main points and conclusions of the article
%  Each must be 100 characters or less with no special characters or punctuation and must be complete sentences

\begin{keypoints}
\item We identified relativistic microbursts observed by the SAMPEX satellite and quantified their duration
\item The microburst duration interquartile range is 70-140 ms and shows trends in AE, L-shell, and MLT
\item In MLT, microburst durations are shortest at midnight and longest at noon---a similar trend to chorus element durations.
\end{keypoints}

%% ------------------------------------------------------------------------ %%
%
%  ABSTRACT and PLAIN LANGUAGE SUMMARY
%
% A good Abstract will begin with a short description of the problem
% being addressed, briefly describe the new data or analyses, then
% briefly states the main conclusion(s) and how they are supported and
% uncertainties.

% The Plain Language Summary should be written for a broad audience,
% including journalists and the science-interested public, that will not have 
% a background in your field.
%
% A Plain Language Summary is required in GRL, JGR: Planets, JGR: Biogeosciences,
% JGR: Oceans, G-Cubed, Reviews of Geophysics, and JAMES.
% see http://sharingscience.agu.org/creating-plain-language-summary/)
%
%% ------------------------------------------------------------------------ %%

%% \begin{abstract} starts the second page

\begin{abstract}
In this study we used the Solar Anomalous and Magnetospheric Particle Explorer to identify and quantify the duration of $>1$ MeV electron microbursts. We investigated the microburst duration trends as a function of geomagnetic activity, L-shell, and magnetic local time (MLT)---with the most evident duration trend in MLT. The shortest microbursts, with a median duration around 80 milliseconds, were observed near midnight MLT. Microburst duration increases from midnight and noon MLT, where at noon the median microburst duration doubles to 160 milliseconds. The increasing microburst duration trend in MLT is similar to the whistler mode chorus rising tone element duration, shedding light into the microburst scattering mechanism.
\end{abstract}

\section*{Plain Language Summary}
\noindent \textcolor{blue}{Finish this section.} Microbursts are a naturally occurring form of electron precipitation from near-Earth space into the atmosphere. They are characterized by their short duration, often assumed to be less than a second. Microburst impact on the atmosphere includes the degradation of Mesospheric Ozone through the production of Odd Nitrogen and Odd Hydrogen molecules... We don't know the details on how microburst electrons are scattered, but there is evidence that they are scattered by whistler-mode chorus rising tone elements... Talk about duration and how it is a probe into the scattering physics.

\section{Introduction}\label{intro}
Earth's outer radiation belt electron population is in constant flux, governed by many processes that affect charged particles via, for example: radial transport, injections from the magnetotail, magnetopause shadowing, and local heating and loss into Earth's atmosphere due to wave-particle interactions \cite<e.g.>[and references within]{Ripoll2020}. Whistler mode chorus (WMC) is just one type of plasma wave characterized by short ($\approx 100$ ms) rising tone elements, and perform a dual role in electron dynamics: accelerating electrons from 10s of keV to MeV energies, as well as pitch angle scattering electrons into the atmosphere \cite<e.g.>{Li2009b, Thorne2010, Horne2003a, Summers2005}. One form of electron precipitation are microbursts: a transient and intense increase of electrons, with a subsecond duration. Microbursts were first observed by balloons in Earth's upper atmosphere, and later by satellites in low Earth orbit (LEO), and recently at high altitude near the magnetic equator \cite<e.g.>{Anderson1964, Blake1996, Lorentzen2001a, O'Brien2003,Douma2017, Kurita2016, Shumko2018b}.

Microburst electron energies span multiple orders of magnitude from tens of keV observed by, for example, \citeA{Datta1997}; to $>1$ MeV observed by the Solar Anomalous Magnetospheric Particle Explorer (SAMPEX) by \citeA{Blum2015}. Microbursts are predominately observed outside the plasmapause on the radiation belt footprints, $L\approx4-8$, and in the midnight to morning Magnetic Local Times (MLT) ($\approx 0-12$ hours MLT) \cite{Lorentzen2001a, Blum2015, O'Brien2003, Douma2017}. While microbursts are observed under all geomagnetic conditions, \citeA{Douma2017} showed that microburst frequency dramatically increases with the Auroral Electrojet (AE) index, and \citeA{O'Brien2003} showed a similar trend with the microburst frequency with the Disturbance storm time index.

The relative impact of microbursts on the ionization of Earth's atmosphere and the depletion of radiation belt electrons is uncertain \textcolor{red}{cite Kathy's paper once it shows up online \cite{Duderstadt2021}}, but the impact of microbursts alone is estimated to be substantial. Microbursts can deplete the outer radiation belt electrons in as little as a few hours, and can deplete up to 20\% of upper mesospheric ozone \cite{O'Brien2004, Thorne2005, Douma2019,Breneman2017,Seppala2018}.

Electron microbursts are widely believed to be scattered by chorus waves. They were associated early on, due to the similar duration of microbursts and chorus rising tone elements, and a similar occurrence distributions in MLT \cite<e.g.>{Lorentzen2001a}. \citeA{Breneman2017} directly linked a chorus rising tone element to a microburst observed by the Focused Investigation of Relativistic Electron Bursts: Intensity, Range, and Dynamics CubeSats (FIREBIRD-II; \citeA{Crew2016, Johnson2020}) during a close magnetic conjunction. \textcolor{blue}{Drop this or add somewhere else: With this evidence, the particle precipitation community is largely in agreement that chorus waves scatter microbursts. \textcolor{blue}{Reword} However, there are other hypothesized drivers such as electromagnetic ion cyclotron \cite{Omura2013, Douma2018}.} 

A natural follow-on question is how are microbursts scattered. For example, it is still unclear if relativistic ($>1$ MeV) microbursts are scattered via cyclotron resonance at high magnetic latitudes, or a higher resonance harmonic near the magnetic equator \cite{Lorentzen2001a}. One way to address this question is by studying for how long microburst electrons are in resonance with a chorus wave. The resulting microburst duration, i.e. the microburst width in the time series data, is a probe into the conditions necessary to scatter microburst electrons. \textcolor{blue}{From our review, we found only qualitative estimates of the microburst duration.} Therefore, we used the microbursts observed by the SAMPEX satellite to quantify the distribution of relativistic microburst duration. In this letter, we quantify the duration distribution of microbursts as a function of L-shell, MLT, and the Auroral Electrojet. We then compare these results to prior chorus rising tone element studies, and a chorus-electron scattering model. 

\section{Instrumentation}\label{instrumentation}
In this study we used the $>1$ MeV electron count data, taken by the Heavy Ion Large Telescope (HILT) instrument \cite{Klecker1993}, onboard the SAMPEX satellite \cite{Baker1993}.

SAMPEX was launched in July 1992 and reentered Earth's atmosphere in November 2012. It was in a 520x670 km, $82^\circ$ inclination low Earth orbit. In general, SAMPEX had two pointing modes: spin and orbit rate rotation (zenith pointing) modes. To avoid the compounding effects due to the variable pitch angles sampled in the spin mode, we only used the zenith pointing mode data. The International Geomagnetic Reference Field \cite[IGRF]{Thebault2015} magnetic field model was used to derive the meomagnetic coordinates.

We used the HILT electron data, sampled at a 20 ms cadence (state4 in the data archive), taken between 1997 and 2012. The HILT instrument consisted of a large rectangular chamber with the aperture on one end, and 16 solid state detectors on the other. The electron counts accumulated over 20 ms were summed from all of the solid state detectors and used in this study.

\section{Methodology}\label{methodology}

We first identified microbursts. Then we fit every microburst's time series to a model, consisting of a Gaussian superposed with a straight line, to quantify the duration for each microburst.

\subsection{Microburst Identification}\label{microburst_id}
We identified microbursts using the burst parameter defined by \citeA{O'Brien2003} and used in numerous other microburst studies with SAMPEX \cite<e.g.>{Douma2017}. Assuming Poisson probability for the observed electron counts, the burst parameter is the number of standard deviations of a foreground signal above the background, expressed as
\begin{equation}
n_\sigma = \frac{N-A}{\sqrt{A+1}}
\end{equation} where $N$ is the number of foreground electron counts (microburst or otherwise), and $A$ is the centered running average background counts. The $1$ in the denominator prevents a division by 0 error. In \citeA{O'Brien2003}, and in the results in this study, $N$ was summed over 100 ms and is called $N_{100}$, while $A$ was summed over 500 ms and is similarly called $A_{500}$.  Henceforth, we specify the time windows with subscripts for $N$ and $A$. Times when $n_\sigma > 10$ are classified as burst times, and the time at the peak count rate in each continuos burst time interval is added to the microburst data set. With $A_{500}$ and $N_{100}$, we detected a total of 256,764 microbursts over the 15 year period from 1997 to 2012. Four examples of microbursts are shown in Fig. \ref{fig1} by the solid black curves.

\begin{figure}
\noindent\includegraphics[width=\textwidth]{figures/fig1.pdf}
\caption{Examples of relativistic microbursts are shown by the black lines, and the fits are shown by the dashed red lines. The fit's full width at half maximum (FWHM) and the $\bar{R}^2$ goodness of fit metric is annotated in each panel. Microbursts with $\bar{R}^2 > 0.9$ were used for this study. The major time ticks are at every second, while the minor ticks are at every 100 milliseconds.}
\label{fig1}
\end{figure}

\subsection{Fitting to Quantify Microburst Duration}
We estimated the microburst duration using two methods that yielded similar results: the duration at half of the microburst's topographic prominence and the full width at half maximum (FWHM) from a Gaussian fit.

The topographic prominence is a simple and robust method to estimate the microburst duration used to identify curtains, a similar-looking type of precipitation \cite{Shumko2020b}. It is defined as the duration at half of the microburst's topographic prominence: the height of the microburst relative to the maximum of the two minima on either side of the microburst peak. On each side of the microburst peak, the minima are searched for between the microburst and a higher peak on that side. While the topographic prominence method of estimating microburst durations is simple and robust, one of its downsides is its inability to automatically verify that the duration is representative of a single microburst. Therefore, we also fit microbursts with a Gaussian, and used the $R^2$ goodness of fit metric to filter out bad duration estimates. By screening out bad fits, we exclude superposition of multiple microbursts that will bias our microburst duration estimates.

We assumed a Gaussian superposed with a straight line fit model. The Gaussian models the shape of the microburst; while the linear trend accounts for the background electrons that are either trapped or quasi-trapped in the drift loss cone. The fit model is defined as:
\begin{equation}
c(t | A, t_0, \sigma, c_0, c_1) = A e^{-\frac{(t-t_0)^2}{2\sigma^2}} + c_0 + c_1 t
\end{equation} where $A$, $t_0$, and $\sigma$ are the Gaussian amplitude, center time, and standard deviation; while the $c_0$ and $c_1$ are the background count intercept and slope. The fit was applied over a number of data points determined by the maximum of either: 4x topographic prominence width or 500 ms. A challenge to any robust and automated nonlinear regression algorithm is guessing the initial parameters. The initial parameter guesses for the Gaussian are provided by the topographic prominence and topographic duration estimates. The two linear trend initial parameters were: $c_0=\mathrm{median(counts)}$ and $c_1=0$. The optimal fit parameters were found using scipy's \url{curve_fit()} function in Python. We defined the microburst duration as the FWHM of the microburst peak, defined by
\begin{equation}
\mathrm{FWHM} = 2\sqrt{2 \ln{2}} \sigma.
\end{equation}

To evaluate the fit, we used the $R^2$ goodness of fit metric. $R^2$ is defined as
\begin{equation}
R^2 = 1 - \frac{SS_{res}}{SS_{mean}} = 1 - \frac{\sum{(c_i-f_i)^2}}{\sum{(c_i-\bar{c})^2}}
\end{equation} where $SS_{res}$ is the sum of the squared residuals between the observed counts $c_i$ and the fit counts $f_i$ for each time step, and likewise $SS_{mean}$ is the sum of the squared residuals between $c_i$ and the mean of the counts, $\bar{c}$.

One interpretation of $R^2$ is: fractionally how much better is the variance in the data explained by the model fit, compared to the null hypothesis horizontal line at $\bar{c}$. $R^2$ values vary from $1$ when the fit perfectly describes the variance in the data, to $-\infty$ for poor fits (a fit can be much worse than the mean null hypothesis).

To account for overfitting that results from the variable number of data points used for each fit, the adjusted $R^2$, $\bar{R}^2$, was used. It is defined as

\begin{equation}
\bar{R}^2 = 1 - (1-R^2) \frac{n-1}{n-p-1}
\end{equation} where $n$ is the number of data points fit, and $p$ is the number of parameters. Intuitively, $n-1$ is the number of degrees of freedom for the null hypothesis, and $n-p-1$ is the degrees of freedom for the fit model. Fits with $\bar{R}^2 > 0.9$ are considered good fits and are used for the rest of this analysis. As a check, we compared the microburst duration estimated with the prominence and fit methods. \textcolor{blue}{Does the following sentence make more sense now?} We chose an agreement criterion between the two methods as durations within 25\%; together with the $\bar{R}^2 > 0.9$ constraint, $85\%$ of microbursts satisfied this agreement criterion.

Figure \ref{fig1}a shows an example of two superposed microbursts that had a fit $\bar{R}^2 = 0.83$ that were excluded from this study. On other hand, Fig. \ref{fig1}b-d show microbursts that were included in this study because the fit $\bar{R}^2 > 0.9$.

Lastly, Fig. \ref{fig1}c,d demonstrate the necessity of the linear fit to account for the changing background. The linear fit accounts for the non-zero mean background counts and the different amplitudes of the edges of the Gaussian. Of the 256,764 detected microbursts, 109,231 had $\bar{R}^2 > 0.9$ and are used for the remainder of this study.

\section{Results}\label{results}
We used the well-fit microbursts to quantify the distribution of microburst duration (FWHM) as a function of the Auroral Electrojet index, L-shell and MLT. We begin with the overall microburst distribution.

Figure \ref{fig2}a shows the distribution of all well-fit microbursts. This distribution is peaked with the median at 96 ms and quickly drops off. The interquartile range spans about a factor of two in microburst duration, from 65 to 139 ms. \textcolor{blue}{A new sanity check!} As a check, we quantified the variation in the percentiles, annotated in Fig. \ref{fig2}a, under varying $\bar{R}^2$ criteria. We found that the percentile variation is minor---the median discrepancy between $\bar{R}^2 > 0.5$ and $\bar{R}^2 > 0.9$ thresholds is $4$ ms.

We then investigated the dependence of microburst duration as a function of geomagnetic activity. To be consistent with the prior wave and microburst studies, we use the AE index to quantify the level of geomagnetic disturbance. We adopt the same three AE intensity levels used in prior studies, such as \citeA{Li2009a}, \citeA{Douma2017}, and \citeA{Meredith2020}: $\mathrm{AE} < 100$, $100 < \mathrm{AE} < 300$, and $\mathrm{AE} > 300$, in units of nanotesla. Figure \ref{fig2}b shows the distribution of microburst duration for the three AE categories. The distributions are qualitatively similar, gradually narrowing and shifting to shorter durations with increasing AE. The median microburst duration decreases from 130 ms for $\mathrm{AE} < 100$ to 95 ms for $ \mathrm{AE} > 300$.

\begin{figure}
\noindent\includegraphics[width=\textwidth]{figures/fig2.pdf}
\caption{Panel a shows the distribution of all microburst full width at full maximum (FWHM). Panel b shows the distribution of all microbursts, categorized by the Auroral Electroject (AE) index into three bins: $\mathrm{AE} < 100$, $100 < \mathrm{AE} < 300$, and $\mathrm{AE} > 300$, in units of nanotesla. The median microburst duration is 130 ms for the $\mathrm{AE} < 100$ ($2.4\times 10^{3}$ microbursts), 111 ms for the $100 < \mathrm{AE} < 300$ ($1.8\times 10^{4}$ microbursts), and 95 ms for the $ \mathrm{AE} > 300$ ($9.3\times 10^{4}$ microbursts) bins.}
\label{fig2}
\end{figure}

Lastly, Figs. \ref{fig3} and \ref{fig4} show the microburst duration as a function of L and MLT. Figure \ref{fig3}a-c shows the joint distributions, split up into the 25th, 50th, and 75th percentiles; Figure \ref{fig4} shows the marginalized distributions as a function of L or MLT.

Figure \ref{fig3} shows that the microburst duration trend is nearly identical for the different percentiles, and thus for simplicity we focus on the median distribution in Fig. \ref{fig3}b. In MLT, the median microburst duration increases by roughly a factor of two: from 80 ms at midnight to 160 ms at noon. In L-shell, the median microburst duration faintly increases with L-shell, and is most apparent near midnight MLT.

To disentangle the L and MLT distributions, Fig. \ref{fig4} shows the marginalized distributions; MLT was marginalized out in Fig. \ref{fig4}a and L-shell was marginalized out in Fig. \ref{fig4}b. Figure \ref{fig4}a shows a slight broadening of the microburst duration at higher L-shells---in contrast to Fig. \ref{fig4}b that clearly shows that the microburst duration increases from midnight to noon MLT.

\begin{figure}
\noindent\includegraphics[width=\textwidth]{figures/fig3.pdf}
\caption{The joint distributions of microburst duration (FWHM) as a function of L-Shell and MLT. Panels a-c show the 25th, 50th, and 75th duration percentiles in each L-MLT bin. The white bins in panels a-c have a statistically insufficient number of microbursts (less than 100). Lastly, panel d shows the distribution of the number of microbursts, with the white bins containing 0 microbursts.}
\label{fig3}
\end{figure}

\begin{figure}
\noindent\includegraphics[width=\textwidth]{figures/fig4.pdf}
\caption{The marginalized distributions of the number of microbursts as a function of microburst duration (FWHM) and L shell in panel a and MLT in panel b.}
\label{fig4}
\end{figure}

\section{Discussion and Conclusions}\label{discussion}

We first discuss a possibility that the burst parameter is less sensitive to microbursts with longer durations and is therefore not detecting them. Recall from Section \ref{microburst_id} that $A$ is the running average counts, centered on the foreground counts $N$, and the burst parameter, $n_\sigma \sim N - A$. Now consider the following hypothesized scenario. Given a microburst with a 500 ms duration and the burst parameter centered on the peak, $A_{500}$ completely overlaps with the microburst and is therefore the mean microburst counts. Then, $n_\sigma$ is proportional to the difference between the mean and the maximum microburst amplitude. However, if we use $A_{1000}$, the mean background counts no longer overlap with just the microburst, but rather the microburst and the lower surrounding background. The resulting $A_{1000}$ is lower than $A_{500}$ and thus the $A_{1000}$ burst parameter is more sensitive to the microburst.

To test this possible bias, we ran the detection algorithm with three background values: $A_{500}$, $A_{1000}$, and $A_{2000}$ and compared the resulting median distributions. The maximum discrepancy in the median microburst duration using the three resulting data sets was 20 ms (one data point); the longer median duration from a longer $A$. This is a 20\% relative discrepancy. Consequently, considering this bias and the distribution in Fig. \ref{fig2}a, the evidence supports that the majority of $>1$ MeV microbursts have a true duration around 100 ms and the $A_{500}$ is adequate to identify them. With more confidence in the detection algorithm, we now shift to the global distribution of microburst durations.

The microburst duration trend in L-shell is subtle; Fig. \ref{fig3}c, and Fig. \ref{fig4}a most clearly show longer durations at higher L-shells near midnight MLT. In contrast, the duration trend in MLT is significant. The median microburst duration doubles from 80 to 160 ms between midnight and noon MLT. Now we will focus on the MLT trend and look for a possible explanation.

As mention in the introduction, chorus rising tone elements are widely believed to scatter microburst electrons \cite<e.g.>{Breneman2017, Saito2012, Miyoshi2020}. Thus, we will compare the microburst duration and chorus trends in local time. Recent studies by \citeA{Teng2017} and \citeA{Shue2019} quantified the properties of equatorial lower band (0.1-0.5 x electron gyrofrequency) chorus rising tone elements. Both studies found that the rising tone element duration distribution peaks at $\approx 250$ ms around midnight, and this peak broadens and shifts to $\approx 500$ ms at noon MLT. The microburst duration and chorus rising tone element trends in MLT scale by roughly the same factor of 2, but in general the chorus rising tone element duration is roughly 3 times longer than $>1$ MeV microbursts. However, theory points to a different chorus property controlling the microburst duration.

Figure 4 in \citeA{Chen2020} shows the result of a microburst test particle simulation. The authors described what wave properties bound the microburst duration in time-energy space. Medium energy ($\approx 50-300$ keV) microburst duration is controlled by the rising tone element bandwidth. Moreover, higher energy microburst duration is controlled by the wave's lower frequency and the upper magnetic latitude propagation. This is in qualitative agreement with the cyclotron resonance condition described in \citeA{Lorentzen2001a}, and the electron time of flight described by \citeA{Saito2012}.

While different model configurations may change what wave properties are theoretically responsible for scattering $>1$ MeV microburst electrons, it is worth noting that Figs. 4 and 5 in \citeA{Shue2019} do not show a clear shift in chorus bandwidth between midnight and noon MLT. Presently, the theory does not conclusively predict what chorus wave properties control the $> 1$ MeV microburst duration, but the chorus rising tone duration trend in MLT is worth further consideration.

As for the AE trend, the median microburst duration decreases from 130 ms, for $AE < 100$ nT, to 95 ms for $AE > 300$ nT. The chorus rising tone duration trend, quantified by \citeA{Teng2017}, is similar: it is broad and peaks at $\approx 500$ ms when $AE < 100$ nT, then narrows and shifts to $\approx 250$ ms when $AE > 300$ nT. While both tend to become shorter with increased AE, the scaling factors are different.

Lastly, high latitude chorus waves, found at $\approx 10-25$ degrees magnetic latitude off of the equator, can also play at important role at scattering microburst electrons \cite{Lorentzen2001a}. \citeA{Li2009a} found that the majority of high latitude chorus waves are constrained to 6-12 MLT. Thus, it is tempting to conclude that the microburst duration trend in MLT can be attributed to the low and high latitude chorus waves. However, because low latitude chorus waves are also observed at 0-12 MLT \cite{Li2009a}, the resulting microburst duration distribution would reflect the chorus wave superposition in the 6-12 MLT region. If low and high latitude chorus waves scattered microbursts with different durations, Fig. \ref{fig4}b would show the durations broaden or bifurcate from midnight to noon MLT. Because Fig. \ref{fig4}b shows the microburst duration only shift to longer durations, high vs low latitude chorus waves are an unlikely explanation for the microburst duration trend in MLT.

In summary, we found that the relativistic microburst duration distribution is peaked at 100 ms, with 75\% of microbursts narrower than 140 ms. We found no significant trend in the microburst duration as a function of L-shell, but we did find a strong trend as a function of MLT---the median microburst duration roughly doubles from 80 ms at midnight, to 160 ms at noon MLT. We found that the microburst duration trend in MLT scales similarly to the rising tone element duration, but the rising tone element duration is longer. Nonetheless, at this time theory is inconclusive: it does not predict that the relativistic microburst duration is controlled by the rising tone element duration, but rather by the chorus bandwidth and/or the upper latitude of propagation.

%%%%%%%%%%%%%%%%%%%%%%%%%%%%%%%%
%% Optional Appendix goes here
%
% The \appendix command resets counters and redefines section heads
%
% After typing \appendix
%
%\section{Here Is Appendix Title}
% will show
% A: Here Is Appendix Title
%
%\appendix
%\section{Here is a sample appendix}

\acknowledgments
We are thankful for the engineers and scientists who made the SAMPEX mission possible. M. Shumko acknowledges the support provided by the NASA Postdoctoral Program at the NASA’s Goddard Space Flight Center, administered by Universities Space Research Association under contract with NASA; L. Blum acknowledges the Heliophysics Innovation Fund program at NASA’s Goddard Space Flight Center; and \textcolor{red}{Alex's funding sources} The SAMPEX HILT and attitude data are located at \url{http://www.srl.caltech.edu/sampex/DataCenter/data.html} and the minute cadence Auroral Electroject data is located at \url{ftp://ftp.ngdc.noaa.gov/STP/GEOMAGNETIC_DATA/INDICES/AURORAL_ELECTROJET/ONE_MINUTE/}.
This analysis software is available at: \url{https://github.com/mshumko/sampex_microburst_widths}, and is archived at \textcolor{red}{Zenodo link (Clean up and document repo before v1.0 release is linked to Zenodo)}. 


%% ------------------------------------------------------------------------ %%
%% References and Citations

%%%%%%%%%%%%%%%%%%%%%%%%%%%%%%%%%%%%%%%%%%%%%%%
%
% \bibliography{<name of your .bib file>} don't specify the file extension
%
% don't specify bibliographystyle
%%%%%%%%%%%%%%%%%%%%%%%%%%%%%%%%%%%%%%%%%%%%%%%

\bibliography{refs.bib}



%Reference citation instructions and examples:
%
% Please use ONLY \cite and \citeA for reference citations.
% \cite for parenthetical references
% ...as shown in recent studies (Simpson et al., 2019)
% \citeA for in-text citations
% ...Simpson et al. (2019) have shown...
%
%
%...as shown by \citeA{jskilby}.
%...as shown by \citeA{lewin76}, \citeA{carson86}, \citeA{bartoldy02}, and \citeA{rinaldi03}.
%...has been shown \cite{jskilbye}.
%...has been shown \cite{lewin76,carson86,bartoldy02,rinaldi03}.
%... \cite <i.e.>[]{lewin76,carson86,bartoldy02,rinaldi03}.
%...has been shown by \cite <e.g.,>[and others]{lewin76}.
%
% apacite uses < > for prenotes and [ ] for postnotes
% DO NOT use other cite commands (e.g., \citet, \citep, \citeyear, \nocite, \citealp, etc.).
%

\end{document}